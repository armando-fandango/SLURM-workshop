\documentclass[10pt,compress]{beamer}
\usetheme{Berlin}
% other themes: AnnArbor, Antibes, Bergen, Berkeley, Berlin, Boadilla, boxes, CambridgeUS, Copenhagen, Darmstadt, default, Dresden, Frankfurt, Goettingen,
% Hannover, Ilmenau, JuanLesPins, Luebeck, Madrid, Maloe, Marburg, Montpellier, PaloAlto, Pittsburg, Rochester, Singapore, Szeged, Warsaw classic

\usecolortheme{seahorse}
% color themes: albatross, beaver, beetle, crane, default, dolphin, dov, fly, lily, orchid, rose, seagull, seahorse, sidebartab, structure, whale, wolverine

%\usefonttheme{serif}
% font themes: default, professionalfonts, serif, structurebold, structureitalicserif, structuresmallcapsserif

\usepackage[utf8x]{inputenc}
\usepackage{ucs}
\usepackage{amsmath}
\usepackage{amsfonts}
\usepackage{amssymb}
\usepackage{graphicx}
\usepackage{animate}

\hypersetup{pdfpagemode=FullScreen} % makes your presentation go automatically to full screen

%\title{\bf HPC User Group of Orlando\\~}
%\subtitle{\textnormal{Introduction to SLURM}}
%\author[Wiegand]{\scriptsize %
%    R. Paul Wiegand\\
%   Institute for Simulation \& Training\\
%    University of Central Florida\\
%    \href{mailto:wiegand@ist.ucf.edu}{wiegand@ist.ucf.edu}
%} 



\author[Armando, Paul, Nicholas]{ \textbf{Armando Fandango} \\R. Paul wiegand \\Nicholas Lucas}

\title[SLURM]{\textbf{Let us learn SLURM}}
\subtitle{A Hands-on Workshop for SLURM on UCF ARCC Clusters}
%\setbeamercovered{transparent} 
%\setbeamertemplate{navigation symbols}{} 
%\logo{} 
\institute[ARCC, UCF]{
  Advanced Research Computing Center \\
  Institute for Simulation \& Training \\
  University of Central Florida
} 
\date{January 2015}

%\subject{subject} 

\setbeamertemplate{footline}
{
  \leavevmode%
  \hbox{%
  \begin{beamercolorbox}[wd=.4\paperwidth,ht=2.25ex,dp=1ex,center]{author in head/foot}%
    \usebeamerfont{author in head/foot}\insertshortauthor\ ( \insertshortinstitute )
  \end{beamercolorbox}%
  \begin{beamercolorbox}[wd=.6\paperwidth,ht=2.25ex,dp=1ex,center]{title in head/foot}%
   \usebeamerfont{title in head/foot}\insertshorttitle\hspace*{3em}
  \end{beamercolorbox}}%
  \vskip0pt%
}
\makeatletter
    \newenvironment{withoutheadline}{
        \setbeamertemplate{headline}[default]
        \def\beamer@entrycode{\vspace*{-\headheight}}
    }{}
\makeatother

\begin{document}

%\frame{\titlepage}

\begin{frame}[plain]
\titlepage
\end{frame}

%\section[Outline]{}
%\frame{\tableofcontents}
\begin{withoutheadline}
\begin{frame}{Topics for Today}
\tableofcontents
\end{frame}
\end{withoutheadline}

\section{Introduction}
\subsection*{}

\begin{frame}[<+->]{What is SLURM?}
\begin{itemize}\setlength{\itemsep}{1.5em}%\small
  \item \textbf{S}imple \textbf{L}inux \textbf{U}tility for \textbf{R}esource \textbf{M}anagement 
  \item 2-in-1: resource manager and scheduler, and so replaces Torque \& Moab
  \item Specially designed for Linux clusters 
  \item Designed to be portable, scalable, fault-tolerant, and \emph{simple}
  \item Open-Source : \url{git://github.com/SchedMD/slurm.git}	
  \item Widely adopted by HPCs around the country (and the world)
\end{itemize}
\end{frame}

\begin{frame}[<+->]{What is the change for STOKES Users?}
\begin{itemize}\setlength{\itemsep}{1.5em}%\small
  \item All basic scheduler commands are different
  \item Submit script directives \& syntax must be changed \\ \bigskip
  {\small\color{red} \textit{Yes, you will have to rewrite your submit scripts}}
  \item That's what we are going to learn today \\ \bigskip
  \textit{SLURM commands and Scripts} :-) 
\end{itemize}
\end{frame}


\begin{frame}{}
 
 \begin{columns}[c]
    \column{.7\textwidth}
        \begin{center}
            \includegraphics[width=\textwidth]{images/sleepy-lecture.png} 
        \end{center}
        \pause
    \column{.3\textwidth}
        \begin{center}
             \includegraphics[width=\textwidth]{images/cat-on-computer.jpg}\\
             So.. Open your computers and login to stokes.
        \end{center}
\end{columns}

\end{frame}


\section{Commands}
\subsection{How to Run Jobs}

\begin{frame}{How to run Jobs?}

%\includegraphics[scale=1]{images/run-run.gif} 


\begin{columns}[c]
    \column{.5\textwidth}
        \begin{center}
          \begin{itemize} \setlength{\itemsep}{1.5em}%\small
  \item srun
  \item sbatch
  \item salloc
\end{itemize}
        \end{center}
    \column{.5\textwidth}
        \begin{center}
             \animategraphics[loop,autoplay,width=\textwidth]{16}{images/run--}{0}{288}
        \end{center}
\end{columns}

\end{frame}

\begin{frame}{srun}
\begin{itemize} \setlength{\itemsep}{1.5em}%\small
  \pause
  \item Try this: \texttt{srun hostname}
  \pause
  \item Try this:  \texttt{srun -N 2 hostname}
  \pause
  \item Try this:  \texttt{srun -n 2 hostname}
  \pause
  \item Try this:  \texttt{srun -N 2 -n 2 hostname}
  \pause
  \item Try this:  \texttt{srun -N 2 -n 4 hostname}
  \pause
  \item  \texttt{-N} or \texttt {--nodes} : number of nodes
  \pause
  \item  \texttt{-n} or \texttt{--ntasks} : number of tasks
  \pause
  \item Best Practice: use either {-n or -N} with --ntasks-per-node
  \end{itemize}
\end{frame}


  \begin{frame}{srun - more options}
\begin{itemize} \setlength{\itemsep}{1.5em}%\small
  \pause
  \item  \texttt{--label} : prepend task-id to output
  \pause
  \item Try this:  \texttt{srun -N 2 -n 4 --label hostname}
  \pause
  \item  \texttt{--ntasks-per-node} : number of tasks per node
    \pause
  \item \texttt{ -c} or \texttt{--cpus-per-task} : number of cores per task
    \pause
  \item \texttt{ -w} or \texttt{--nodelist} : specific nodes to be allocated
    \pause
  \item \texttt{ -x} or \texttt{--exclude} : specific nodes not to be allocated
\end{itemize}
\end{frame}

\begin{frame}{srun - even more options}
\begin{itemize} \setlength{\itemsep}{1.5em}%\small
  \pause
  \item \texttt{--time}
  \pause  
  \item \texttt{--account}
    \pause
  \item \texttt{--output}
    \pause
  \item \texttt{--job-name}
    \pause
  \item \texttt{--mem}
    \pause
  \item \texttt{--mem-per-cpu}
    \pause
  \item \texttt{man srun} or \texttt{srun --help} :  to read manual for more options  

\end{itemize}
\end{frame}

\begin{frame}{salloc}
\begin{itemize} \setlength{\itemsep}{1.5em}%\small
  \pause
  \item Try this:  \texttt{salloc hostname}
  \pause
  \item Try this:  \texttt{salloc}
  \pause
   \item Try this: \texttt{srun hostname}
   \pause
  \item salloc is almost same as srun !!!!
  \pause
  \item salloc only reserves nodes and starts a shell
  \pause
  \item srun runs the --number-of-tasks-- copies of your app
\end{itemize}
\end{frame}

\subsection{How to get Interactive Session}

\begin{frame}{interactive sessions with srun}
\begin{itemize} \setlength{\itemsep}{1.5em}%\small
  \pause
  \item Try this: \texttt{srun --pty bash}
  %\pause
   %\item For X-window: \texttt{srun --x11} // X11 option not yet implemented on stokes.. but keep an eye for future.
   
\end{itemize}
\end{frame}

\begin{frame}{interactive sessions with salloc}
\begin{itemize} \setlength{\itemsep}{1.5em}%\small
   \item Get allocation and note job-id: \texttt{salloc}
   \pause
   \item Get node: \texttt{squeue --job <job-id>}
   \pause
   \item ssh into node. \texttt{ssh ec<node-id>}
   \pause
   \item load module: \texttt{module load starccm}
   \pause
   \item run: \texttt{starccm+}
   \pause
   \item ssh into node. \texttt{ssh -X ec<node-id>}
   \pause
   \item load module: \texttt{module load starccm}
   \pause
   \item run: \texttt{starccm+}
      
\end{itemize}
\end{frame}


\subsection{How to Monitor Jobs}
\begin{frame}{squeue}
\begin{itemize} \setlength{\itemsep}{1.5em}%\small
  \pause
  \item Open two terminal windows side by side and try following commands
  \pause
  \item Terminal Window 1
  \begin{itemize}
    \item \texttt{watch "squeue -t all"}
  \end{itemize}  
  \pause
  \item Terminal Window 2
   \begin{itemize}
     \item \texttt{srun -N 2 sleep 60 \&}
      \item \texttt{srun -N 4 uptime}
       \item \texttt{srun -N 2 hostname}
       \end{itemize}
  \pause
  \item Do you see the jobs come and go in Window 1 ?
\end{itemize}
\end{frame}


\subsection{How to Get Information}

\begin{frame}{scontrol show}
\begin{itemize} \setlength{\itemsep}{1.5em}%\small
  \pause
  \item Try this:  \texttt{scontrol show partition}
  \pause
  \item Try this: \texttt{scontrol show job}
  \pause
  \item  Run this: \texttt{srun -N 2 sleep 60 \&}
  \pause
  \item  Get the job id: \texttt{squeue --user=`whoami`}
  \pause
  \item  Now run this: \texttt{scontrol --detail show job <job-id>}
\end{itemize}
\end{frame}

\begin{frame}{sinfo}
\begin{itemize} \setlength{\itemsep}{1.5em}%\small
  \item a "queue" is called \textit{partition} in SLURM
  \pause
  \item Try this : \texttt{sinfo}
  \pause
  \item Try this:  \texttt{sinfo -all}
  \pause
  \item Try this: \texttt{sinfo -Node}
\end{itemize}
\end{frame}

\begin{frame}{sacct}
\begin{itemize} \setlength{\itemsep}{1.5em}%\small
  \pause
  \item  Try this: \texttt{sacct}
  \pause
  \item  Try this: \texttt{sacct -j <job-id>}
  \pause
  \item  Try this: \texttt{sacct  --long  -j <job-id>}
\end{itemize}
\end{frame}

\begin{frame}{sshare \& sreport}
\begin{itemize} \setlength{\itemsep}{1.5em}%\small
  \pause
  \item  Try this: \texttt{sshare}
  \pause
  \item  Try this: \texttt{sshare -l}
   \pause
  \item  Try this: \texttt{sreport -a cluster AccountUtilizationByUser Tree}
\end{itemize}
\end{frame}



\subsection{How to Manage Jobs}

\begin{frame}{scancel}
\begin{itemize} \setlength{\itemsep}{1.5em}%\small
  \pause
  \item  Run this: \texttt{srun -N 2 sleep 60 \&}
  \pause
  \item  Get the job id: \texttt{squeue --user=`whoami`}
  \pause
  \item  \texttt{scancel <job-id>}
  \pause
  \item  Now run this: \texttt{squeue --user=`whoami`}
\end{itemize}
\end{frame}

\begin{frame}{scontrol}
\begin{itemize} \setlength{\itemsep}{1.5em}%\small
  \pause
  \item  Run this: \texttt{srun -N 2 sleep 60 \&}
  \pause
  \item  Get the job id: \texttt{squeue --user=`whoami`}
  \pause
  \item  \texttt{scontrol hold <job-id>}
  \pause
  \item  Now run this: \texttt{squeue --user=`whoami`}
   \pause
  \item  \texttt{scontrol release <job-id>}
  \pause
  \item  Now run this: \texttt{squeue --user=`whoami`}
  \pause
  \item  \texttt{scontrol requeue <job-id>}
  \pause
  \item  Now run this: \texttt{squeue --user=`whoami`}
\end{itemize}
\end{frame}


\section{Scripts}
\subsection{How to write Batch Job Scripts}
\begin{frame}[fragile]{Batch Job Script}

  Its nothing new !!!! Put together what you learnt so far.
  
  {\scriptsize
   \begin{verbatim}
#!/bin/bash
#SBATCH --account=arcc
#SBATCH --nodes=2
#SBATCH --ntasks-per-node=2
#SBATCH --time=00:10:00
#SBATCH --error=rpwslurm-%J.err
#SBATCH --output=rpwslurm-%J.out
#SBATCH --job-name=PaulSlurmMPIJob
#SBATCH --mail-type=FAIL
#SBATCH --mail-type=BEGIN
#SBATCH --mail-type=END
#SBATCH --mail-user rpwiegand@gmail.com

# Load modules
module load openmpi/openmpi

# Output Information
echo "Slurm nodes: $SLURM_JOB_NODELIST"

# Run your app
mpirun ./hello

  \end{verbatim}
}
\end{frame}

\subsection{How to get NodeList}

\begin{frame}[fragile]{Getting NodeList}
My application requires NODEFILE or MACHINEFILE
\pause
\begin{itemize}
    \item  Get nodes: \verb.salloc -N 2 -n 6. 
 \end{itemize}
  \pause
\bigskip
 {
  \scriptsize
   
  \verb.echo $SLURM_JOB_NODELIST. \\
  \pause
  \verb-ec[3,6]-
  \pause
  \\\ \\
   \verb.export NODELIST=`srun hostname | sort` ; echo $NODELIST. \\
  \pause
  \verb-ec3 ec3 ec3 ec3 ec6 ec6-
  \pause
\\\ \\
     \verb.export NODELIST=`srun hostname | sort -u` ; echo $NODELIST. \\
  \pause
  \verb-ec3 ec6-
  \pause
\\\ \\
     \verb.export NODELIST=`srun hostname | sort -u | sed s/"ec"/"ic"/ ` ; echo $NODELIST.\\
  \pause
  \verb-ic3 ic6-
\pause
\\\ \\
     \verb.export NODELIST=`srun hostname | sort -u | sed s/"ec"/"ic"/ |.
     \verb.                             tr '\n' ',' | sed s/",$"//` ; echo $NODELIST.\\
  \pause
  \verb-ic3,ic6-
  }

  
  

\end{frame}


\section{Conclusion}

\begin{frame}{Final Thoughts}
We hope this will enable you to start writing your own slurm scripts.
\pause \\
\bigskip
Use the SLURM documentation at \url{http://slurm.schedmd.com/}.\\
\pause
\bigskip
Stay Tuned: \\ Advanced SLURM workshop : Job Arrays, Newton (GPGPU and Phi), Job Sequencing, Task-Core-Node Allocation Strategies, and more.\\

\end{frame}
\begin{frame}{Questions}
\center{
\includegraphics[width=\textwidth]{images/questions.jpg} 
}
\end{frame}

\begin{frame}{Thanks}
\center{
Thanks for coming today.
\\ \bigskip
email us: \url{request-stokes@ist.ucf.edu}
}
\end{frame}



%\section{References}


\end{document}